%%=============================================================================
%% Install Mariadb Replication
%%=============================================================================

\section{MariaDB: Replication installation guide}

\subsection{Steps}

\subsubsection{Database replication}

\begin{enumerate}
    \item install MariaDB
    \item start and config MariaDB on primary server
    \item config master firewall
    \item start and config MariaDB on replica server
    \item test MariaDB
\end{enumerate}

\subsubsection{Reverse proxy}

\begin{enumerate}
    \item install proxy
    \item config proxy
    \item test proxy
\end{enumerate}

\subsection{Install MariaDB}

The repository is already there so we just have to install some packages.

\begin{lstlisting}
    sudo dnf install MariaDB-server MariaDB-backup
\end{lstlisting}

\subsection{Start/config primary server}

Edit config file with following:

\begin{lstlisting}
    [mariadb]
    log_bin
    server_id=1
    log-basename=master
    binlog-format=mixed
\end{lstlisting}

\textit{server\_id} is a unique value.\\
Start the dbserver:

\begin{quote}
    If already running, restart dbserver to apply config.
\end{quote}

\begin{lstlisting}
    systemctl enable mariadb --now
\end{lstlisting}

There should be at least 1 user created for replication:

Create user accounts:

\begin{quote}
    Each account should be created on the master so replication can make these accounts on the slaves.
\end{quote}

\subsubsection{Replication user}

\begin{lstlisting}
    CREATE USER 'repl'@'slave_ip' IDENTIFIED BY 'test';
\end{lstlisting}

Grant required privileges.

\begin{lstlisting}
    GRANT REPLICATION SLAVE
    ON *.* TO repl@'slave_ip';
\end{lstlisting}

\subsection{Start/config replica server}

\begin{lstlisting}
    [mariadb]
    log_bin
    server_id=2
    log-basename=master
    binlog-format=mixed
\end{lstlisting}

\textit{server\_id} is a unique value.\\

\begin{lstlisting}
    SHOW SLAVE STATUS;
    +--------------------+----------+--------------+------------------+
    | File               | Position | Binlog_Do_DB | Binlog_Ignore_DB |
    +--------------------+----------+--------------+------------------+
    | master1-bin.000096 |      568 |              |                  |
    +--------------------+----------+--------------+------------------+
\end{lstlisting}

Take notes of the filename and position

\begin{lstlisting}
    CHANGE MASTER TO
    MASTER_HOST='master_ip',
    MASTER_USER='replication_user',
    MASTER_PASSWORD='test',
    MASTER_PORT=3306,
    MASTER_LOG_FILE='master-bin.000096',
    MASTER_LOG_POS=568,
    MASTER_CONNECT_RETRY=10;
\end{lstlisting}

\begin{quote}
    With fresh master, you don't need to specify `MASTER\_LOG\_FILE` and `MASTER\_LOG\_POS`
\end{quote}

\subsubsection{Start replication}

\begin{lstlisting}
    START SLAVE;
    SHOW SLAVE STATUS;
\end{lstlisting}

If replication is running properly, both `Slave\_IO\_Running` and `Slave\_SQL\_Running` should be `Yes`.


\subsection{Test replication}

Use the following url to test replication: \url{https://mariadb.com/docs/deploy/topologies/primary-replica/enterprise-server-10-3/test-es/}

\subsection{Install proxy}

I chose to go with `haproxy`.\\
To install this proxy just use dnf.

\begin{lstlisting}
    sudo dnf install haproxy
\end{lstlisting}

\subsection{Config proxy}

\subsubsection{Edit configuration file}

\begin{lstlisting}
    # /etc/haproxy/haproxy.cfg
    
    defaults
    mode  tcp
    option mysql-check user haproxy_health
    frontend frontend
    # read-requests will arrive at port 3100
    bind *:3100
    # write-requests will arrive at port 3200
    bind *:3200
    # if write send to masters
    use_backend db_masters if { dst_port 3200 }
    # if read sent to slaves
    default_backend db_slaves
    
    backend db_masters
    server master1 10.8.131.120:3306
    
    backend db_slaves
    server slave1 10.8.131.121:3306
    server master1 10.8.131.120:3306
\end{lstlisting}

\subsubsection{(re)start service}

\begin{lstlisting}
    sudo systemctl enable haproxy --now
\end{lstlisting}

\subsubsection{SELinux}

\begin{lstlisting}
    setsebool haproxy_connect_any 1
\end{lstlisting}

\subsection{Test proxy}

To test proxy, use  port 3200 te write data and port 3100 to query data.