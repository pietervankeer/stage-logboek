%%=============================================================================
%% Verklarende woordenlijst
%%=============================================================================
\newglossaryentry{kwartier}
{
    name=kwartier,
    description={Tijdelijke verblijfplaats voor militairen~\autocite{Wikipedia2021}},
    plural=kwartieren
}
\newglossaryentry{ccvc}
{
    name={CC V\&C},
    description={Competentie centrum Vliegend materieel \& Communicatie- en informatie-systemen}
}
\newglossaryentry{its}
{
    name=IT\&S,
    description={Information Technology \& Services}
}
\newglossaryentry{eenheid}
{
    name=eenheid,
    description={Een militaire eenheid bestaat uit een commando- en staf-element, uit uitvoerende militaire eenheden van een lager niveau en steuneenheden. Een militaire eenheid staat onder commando van een (onder)officier van voldoende rang~\autocite{Wikipedia2021a}},
    plural=eenheden
}
\newglossaryentry{lvm}
{
    name=LVM,
    description={Linux Volume Manager}
}
\newglossaryentry{rdp}
{
    name=RDP,
    description={Remote desktop protocol}
}
\newglossaryentry{acl}
{
    name=ACL,
    description={Access control list},
    plural={ACL's}
}
\newglossaryentry{selinux}
{
    name=SELinux,
    description={Security Enhanced Linux}
}
\newglossaryentry{db}
{
    name=db,
    description={Database},
    plural=db's
}
\newglossaryentry{chod}
{
    name=chef Defensie,
    description={De chef Defensie, ook CHOD (Chief of Defence) of stafchef genoemd, is het hoofd van de Belgische Defensie en van de vier componenten. Hij is de hoogste bevelhebber (na de koning) en hoeft alleen aan de bevoegde voogdijminister verantwoording af te leggen.~\autocite{Wikipedia2021b}}
}