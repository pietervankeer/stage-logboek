%%=============================================================================
%% Opdrachten
%%=============================================================================
% Opdracht template:

%\subsubsection{Beginsituatie}
%\subsubsection{Doel}
%\subsubsection{Plan van Aanpak}
%\subsubsection{Uitwerking}
%\subsubsection{Eindresultaat}
%\subsubsection{Business doelstellingen}
%\subsubsection{Persoonlijke doelstellingen}

\section{Opdrachten}
\label{sec:opdrachten}

\subsection{Backup-script moderniseren}

Binnen Defensie gebruikt men een bashscript genaamd ``bu\_script.sh'' om een backup te nemen van de databanken die bestaan op een machine en deze weg te schrijven op het netwerk.  
Ik kreeg de opdracht om dit script te moderniseren. Om de opdracht een beetje te vereenvoudigen moest ik de backup's lokaal wegschrijven. Het script is te vinden in Bijlage \ref{sec:bu_script}

\subsubsection{Beginsituatie}

Het script bestaat maar het is verouderd. Met het script is het mogelijk om meerdere instances van Mariadb/mysql te backuppen. Het script is zeer lang en moeilijk leesbaar.

\subsubsection{Doel}

\begin{itemize}
    \item moderniseer het script
    \item maak het script meer flexibel door gebruik te maken van meer variabelen
    \item genereer een rapport van de uitvoering van het script
    \item schrijf de backup lokaal weg onder de map ``$\sim$/backup''
\end{itemize}

\subsubsection{Plan van Aanpak}

\begin{enumerate}
    \item lees het script door en probeer de denkwijze te begrijpen van de persoon die het script geschreven heeft
    \item verwijder delen die niet meer van toepassing zijn
    \item voeg meer commentaar toe aan het script zodat iemand die het script later zal doornemen sneller weet wat er gebeurd
    \item gebruik meer functies aan zodat het script meer leesbaar wordt
    \item test het script
\end{enumerate}

\subsubsection{Uitwerking}

Ik ben begonnen met het script door te nemen en de logica proberen te verstaan. Tijdens dat ik dit deed heb ik extra commentaar geschreven om duidelijk te maken wat er juist gebeurde in het script. Na een kort gesprek met Tom hebben we besloten om het script enkel te beperken voor server die maar 1 instance hebben van mariadb/mysql. Hierdoor werd het script al veel meer leesbaar. Door extra hulpfuncties toe te voegen is de leesbaarheid ook verbeterd. Tijdens het testen heb ik er nog enkele bugs uitgehaald maar het script werkt zoals gevraagd.

\subsubsection{Eindresultaat}

Het eindresultaat is een script dat makkelijker te lezen/verstaan en flexibel is. Het script gaat een rapport genereren van de uitvoering en de backup was te vinden in de map ``$\sim$/backup''. Er is mogelijkheid om in te toekomst het script uit te breiden zodat dit een melding gaat genereren in Zabbix (monitoring tool).

\subsubsection{Business doelstellingen}

Voor de business was het belangrijk dat het script een moderne update kreeg. Dit was zeker het geval.

\subsubsection{Persoonlijke doelstellingen}

Voor mij persoonlijk was dit een geslaagde opdracht. Het script is nu meer flexibel en dit zal er voor zorgen dat men dit in de toekomst nog kan uitbreiden.


\subsection{Procedure upgrade mariadb}

Tegen 2024 moeten alle dbserver binnen Defensie draaien op Mariadb 10.6 op RHEL 8. Om deze transitie rustig te laten verlopen is er nood aan een procedure die men kan volgen tijdens zo een upgrade van een machine.

\subsubsection{Beginsituatie}

De dbserver bij Defensie draaien op mariadb 5.5 op RHEL 7.

\subsubsection{Doel}

Schrijf een procedure die men kan volgen om een machine te upgraden naar Mariadb 10.6 op RHEL 8. De procedure moet aandacht schenken aan volgende punten:

\begin{itemize}
    \item het besturingssysteem van de nieuwe machine is Red Hat Enterprise Linux (RHEL) 8
    \item er gaat geen data verloren
    \item de juiste mensen inlichten dat er onderhoud zal doorgevoerd worden op hun dbserver
\end{itemize}

\subsubsection{Plan van Aanpak}

Voor deze opdracht zal ik iteratief werken:

\begin{itemize}
    \item de eerste iteratie zal ik op papier (in grote lijnen) schetsen wat er moet gebeuren
    \item de tweede iteratie zal ik dit in een word-document gieten en laten nalezen door Tom
    \item de derde en volgende iteratie(s) zal ik de procedure meer specifiek maken
    \item eens de procedure duidelijk is, ga ik te werk om bepaalde hulpscripts te schrijven. Deze moeten er voor zorgen dat de persoon die de transitie doet zo weinig mogelijk manueel werk heeft.
    \item schrijf het script om users te checken
    \item als laatste zal ik de procedure grondig testen.
\end{itemize}

\subsubsection{Uitwerking}

Nadat de procedure (zie Bijlage \ref{sec:upgrade-mariadb}) opgesteld was en nagekeken door Tom ben ik begonnen aan het script \verb*|check_users| (zie Bijlage \ref{sec:check-users}).

\subsubsection{Eindresultaat}
\subsubsection{Business doelstellingen}
\subsubsection{Persoonlijke doelstellingen}

\subsection{High availability bij databanken}


Onderzoek hoe men binnen defensie high availability implementeren? Als er een db faalt en niet beschikbaar is moet de applicatie zonder problemen verderwerken.

\subsubsection{Beginsituatie}

Binnen Defensie is er momenteel maar 1 databank achter een bepaalde applicatie. Dit zorgt er natuurlijk voor dat als die databank faalt, de applicatie ook niet meer zal werken.

\subsubsection{Doel}

Om de kans op het verliezen van data te verkleinen wil men bij Defensie databank replicatie implementeren.

\subsubsection{Plan van Aanpak}

Voor deze opdracht zal ik iteratief werken:

\begin{itemize}
    \item de eerste iteratie zal ik op papier (in grote lijnen) schetsen wat er moet gebeuren alsook ideeën opschrijven van productie die men mogelijk kan gebruiken
    \item de tweede iteratie zal ik dit in een word-document gieten en laten nalezen door Tom
    \item ik ga een master-slave opstelling testen
    \item ik ga proberen om de master-slave opstelling om te vormen naar een master-master
\end{itemize}

Nadat het document volledig is ga ik een handleiding maken om dit te implementeren en een eenvoudige opstelling testen met drie servers (twee \gls{db} servers en 1 een proxyserver).

\subsubsection{Uitwerking}

Ik heb een installatie handleiding (zie bijlage \ref{sec: replication-installation-guide}) gemaakt om een master-master opstelling op te zetten.

\subsubsection{Eindresultaat}

Er is een handleiding gemaakt die men kan volgen om master-master replicatie op te zetten voor een applicatie. Als er 1 server wegvalt dan zal de proxy automatisch schakelen tussen de \gls{db} servers.

\subsubsection{Business doelstellingen}

Er is replicatie tussen de databank servers.

\subsubsection{Persoonlijke doelstellingen}

Ik heb bijgeleerd over replicatie tussen databanken en hun verschillende opstellingen.


\subsection{MS SQL logins}

\subsubsection{Beginsituatie}

De Jboss Application servers connecteren zich met MSSQL server met SQL logins en dus via SQL server authentication.

\subsubsection{Doel}

Om het systeem veiliger te maken wil de dienst \gls{mssql} overschakelen naar Windows authenticatie bij het inloggen op een databank.

\subsubsection{Plan van Aanpak}

Ik ga beginnen met mezelf bekend te maken met het onderwerp. Ik ga een virtuele machine vragen waar ik op kan testen. Daarna zal ik een oplossing proberen zoeken op het probleem, en als laatste zal ik deze testen.

\subsubsection{Uitwerking}

Ik had een virtuele machine gevraagd die ik kon gebruiken als testomgeving. Blijkbaar is het niet zo simpel om voor mij een virtuele machine te voorzien dus hebben ze mij toegang gegeven op een bestaande databank. Zo had ik een omgeving om dingen te testen. Ik heb een gesprek gehad met Donovan om een oplossing te zoeken op het probleem.

\subsubsection{Eindresultaat}

Het is mogelijk om om windows authenticatie te gaan gebruiken voor de JBOSS applicaties, maar het is niet onbelangrijk om aandacht te schenken aan het volgende. Het wachtwoord van de gebruiker die je in \gls{ad} aanmaakt zal na een bepaalde tijd moeten veranderd worden (afgedwongen door de policy van \gls{ad}), dit is een grote administratieve taak als je dit moet doen voor alle servers. Om aan Windows authenticatie te doen kam men gebruik maken van dezelfde driver dat men nu ook gebruikt.

\subsubsection{Business doelstellingen}

Het systeem zal veiliger zijn met Windows Auhenticatie omdat er dan gebruikers gebruikt worden die gedefinieerd zijn in \gls{ad}.