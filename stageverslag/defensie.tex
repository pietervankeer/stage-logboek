%%=============================================================================
%% Defensie
%%=============================================================================
%% stel stageplaats voor en schets waar je stage hebt gelopen.

\section{Defensie}
\label{sec:defensie}

Defensie, of ook wel het Belgisch leger, staat in voor de bescherming van de strategische belangen van België. De opperbevelhebber van Defensie is koning Filip (zie Figuur \ref{fig:koning_filip}), Minister van Defensie is Ludivine Dedonder (zie Figuur \ref{fig:mod}) en de \gls{chod}, de stafchef van de Generale staf is admiraal Michel Hofman (zie Figuur \ref{fig:chod})~\autocite{Wikipedia2022a}

\begin{figure}
    \includegraphics[width=3cm]{img/koning_filip.jpg}
    \caption{\label{fig:koning_filip}Filip Leopold Lodewijk Maria Van België, de zevende koning der belgen~\autocite{Wikipedia2022}}
\end{figure}

\begin{figure}
    \includegraphics[width=3cm]{img/Ludivine_Dedonder.jpg}
    \caption{\label{fig:mod}Ludivine Dedonder, minister van Defensie~\autocite{Wikipedia2022c}}
\end{figure}

\begin{figure}
    \includegraphics[width=3cm]{img/michel_hofman.jpg}
    \caption{\label{fig:chod}Admiraal Michel Hofman, \gls{chod}~\autocite{Nato2020}}
\end{figure}

%bron: https://www.mil.be/nl/over-defensie/#onze-organisatie
Defensie telt 26.179 personeelsleden, verdeeld over vier componenten: Landcomponent, Luchtcomponent, Marine en Medische Component. Daarnaast werken ze ook op de verschillende algemene directies en stafdepartementen, onder leiding van de \gls{chod}: admiraal Michel Hofman. Zie figuur~\ref{fig:organigram-defensie}

De personeelsleden van Defensie oefenen meer dan 300 verschillende beroepen uit. Dat maakt van Defensie een van de meest diverse werkgevers in het land, met zowel Nederlandstalige (54 \%) als Franstalige (46 \%) profielen. Militairen vormen de bulk van het personeelsbestand. Sommige functies vereisen echter geen uniform: bijna 6 \% ervan wordt ingevuld door burgerpersoneel. \autocite{Defensie2022}

\begin{figure}
    \includegraphics[width=\textwidth]{img/organigram-defensie.png}
    \caption{\label{fig:organigram-defensie}Organigram van Defensie~\autocite{Defensie2022}}
\end{figure}

\subsection{Stageplaats}

Defensie heeft verschillende \glspl{kwartier}, mijn stage vond plaats binnen de \gls{eenheid} \gls{ccvc} op het \gls{kwartier} Majoor Housiau gelegen te Peutie. Zie Figuur~\ref{fig:organigram-ccvc}

\begin{figure}
    \includegraphics[width=\textwidth]{img/organigram-ccvc.png}
    \caption{\label{fig:organigram-ccvc}Organigram van de \gls{eenheid} \gls{ccvc}~\autocite{Defensie2022a}}
\end{figure}

Mijn stage was bij de afdeling Systems binnen het Departement \gls{its}. Zie figuur~\ref{fig:organigram-its}

\begin{figure}
    \includegraphics[width=\textwidth]{img/organigram-its.png}
    \caption{\label{fig:organigram-its}Organigram van het departement \gls{its}~\autocite{Defensie2022a}}
\end{figure}

Binnen de afdeling Systems zijn er 6 diensten:

\begin{itemize}
    \item linux
    \item windows
    \item oracle databanken
    \item Microsoft sql server
    \item prodctl
    \item techbu
\end{itemize}

Mijn stage was binnen de dienst linux, onder leiding van Tom De Leeuw. Hier kwam ik in een team van drie personen die zich bezigheden met het beheer van de linux servers. Specifiek staat dit team in voor de monitoring, provisioning, patching van de linuxservers (databank-, applicatieservers) binnen Defensie.
Ik heb gekozen om bij Defensie stage te lopen omdat ik ambitie heb om later bij Defensie te gaan werken. Via deze weg kan ik al eens proeven hoe het er op de werkvloer aan toe gaat.