\documentclass[a4paper]{article}

% packages
\usepackage[utf8]{inputenc}
\usepackage[dutch]{babel}
\usepackage{hyperref}
\usepackage{graphicx}
\usepackage{listings}
\usepackage[toc]{glossaries}
\usepackage{xcolor}
\usepackage{csquotes}


% bibliografische databank
\usepackage[backend=biber, style=apa]{biblatex}
\DeclareLanguageMapping{dutch}{dutch-apa}
\addbibresource{stageverslag-2122.bib}

% listings settings 
\definecolor{codegreen}{rgb}{0,0.6,0}
\definecolor{codegray}{rgb}{0.5,0.5,0.5}
\definecolor{codepurple}{rgb}{0.58,0,0.82}
\definecolor{backcolour}{rgb}{0.95,0.95,0.92}

\lstdefinestyle{mystyle}{
    backgroundcolor=\color{backcolour},   
    commentstyle=\color{codegreen},
    keywordstyle=\color{magenta},
    numberstyle=\tiny\color{codegray},
    stringstyle=\color{codepurple},
    basicstyle=\ttfamily\footnotesize,
    breakatwhitespace=false,         
    breaklines=true,                 
    captionpos=b,                    
    keepspaces=true,                 
    numbers=left
}


\lstset{style=mystyle}

% glossary settings
\makeglossaries
%%=============================================================================
%% Verklarende woordenlijst
%%=============================================================================
\newglossaryentry{kwartier}
{
    name=Kwartier,
    description={Tijdelijke verblijfplaats voor militairen},
    plural=kwartieren
}
\newglossaryentry{glsy}
{
    name=test,
    description={Acronyms and terms which are generally unknown or new to common readers}
}

% Metadata
\title{Stageverslag}
\author{Pieter {Van Keer}}
\date{2021 - 2022}

\begin{document}
    
    % inhoudstafel
    % set toc depth to only display section and subsections.
    \setcounter{tocdepth}{2}
    \tableofcontents
    \pagebreak
    
    % kern
    %%=============================================================================
%% Voorwoord
%%=============================================================================
\section{Voorwoord}
\label{sec:voorwoord}

%% TODO:
%% Het voorwoord is het enige deel van de bachelorproef waar je vanuit je
%% eigen standpunt (``ik-vorm'') mag schrijven. Je kan hier bv. motiveren
%% waarom jij het onderwerp wil bespreken.
%% Vergeet ook niet te bedanken wie je geholpen/gesteund/... heeft



Dit stageverslag werd geschreven in het kader van de stage die ik liep bij Defensie van 21/02/2022 tot 27/05/2022. Dit was een zeer interessante periode omdat ik hier veel bijgeleerd heb.

Zoals vele andere dingen in het leven, kon ik dit niet alleen en daarom wil ik graag een aantal mensen bedanken. Steven beeckman, bedankt om mij zo goed te ontvangen bij Defensie, ook voor de vragen die ik had voor zowel de stage als bachelorproef kon ik steeds bij jou terecht. Tom De Leeuw, mijn stagementor, bedankt om mij te begeleiden en te helpen doorheen mijn stage. Wim De bruyn, mijn stagebegeleider, bedankt voor de opbouwende feedback en begeleiding tijdens de stage. Tot slot, Britt, die mij ondertussen al 3 jaar aanzet om het beste uit mezelf te halen.

\begin{flushleft}
    
Ik wens u veel leesplezier.
\end{flushleft}

\begin{flushleft}

Pieter Van Keer

Dendermonde, 27 mei, 2022

\end{flushleft}
    %%=============================================================================
%% Defensie
%%=============================================================================
%% stel stageplaats voor en schets waar je stage hebt gelopen.

\section{Defensie}
\label{sec:defensie}
%bron: https://www.mil.be/nl/over-defensie/#onze-organisatie
Defensie telt 26.179 personeelsleden, verdeeld over vier componenten: Landcomponent, Luchtcomponent, Marine en Medische Component. Daarnaast werken ze ook op de verschillende algemene directies en stafdepartementen, onder leiding van de Chef Defensie: admiraal Michel Hofman. Zie figuur~\ref{fig:organigram-defensie}

De personeelsleden van Defensie oefenen meer dan 300 verschillende beroepen uit. Dat maakt van Defensie een van de meest diverse werkgevers in het land, met zowel Nederlandstalige (54 \%) als Franstalige (46 \%) profielen. Militairen vormen de bulk van het personeelsbestand. Sommige functies vereisen echter geen uniform: bijna 6 \% ervan wordt ingevuld door burgerpersoneel. \autocite{Defensie2022}

\begin{figure}
    \includegraphics[width=\textwidth]{img/organigram-defensie.png}
    \caption{\label{fig:organigram-defensie}Organigram van Defensie~\autocite{Defensie2022}}
\end{figure}

\subsection{Stageplaats}

Defensie heeft verschillende \glspl{kwartier}, mijn stage vond plaats binnen de \gls{eenheid} \gls{ccvc} op het \gls{kwartier} Majoor Housiau gelegen te Peutie. Zie Figuur~\ref{fig:organigram-ccvc}

\begin{figure}
    \includegraphics[width=\textwidth]{img/organigram-ccvc.png}
    \caption{\label{fig:organigram-ccvc}Organigram van de \gls{eenheid} \gls{ccvc}~\autocite{Defensie2022a}}
\end{figure}

Mijn stage was bij de afdeling Systems binnen het Departement \gls{its}. Zie figuur~\ref{fig:organigram-its}

\begin{figure}
    \includegraphics[width=\textwidth]{img/organigram-its.png}
    \caption{\label{fig:organigram-its}Organigram van het departement \gls{its}~\autocite{Defensie2022a}}
\end{figure}

Binnen de afdeling Systems zijn er 6 diensten:

\begin{itemize}
    \item linux
    \item windows
    \item oracle databanken
    \item Microsoft sql server
    \item prodctl
    \item techbu
\end{itemize}

Hoewel ik binnen elke dienst een introductie gekregen heb was miin stage vooral binnen de dienst linux.

Ik heb Defensie gekozen omdat ik ambitie heb om later bij Defensie te gaan werken. Op deze manier kan ik al eens kijken hoe het er aan toe gaat op de werkvloer.
    %%=============================================================================
%% Opdrachten
%%=============================================================================
% Opdracht template:

%\subsubsection{Beginsituatie}
%\subsubsection{Doel}
%\subsubsection{Plan van Aanpak}
%\subsubsection{Uitwerking}
%\subsubsection{Eindresultaat}
%\subsubsection{Business doelstellingen}
%\subsubsection{Persoonlijke doelstellingen}

\section{Opdrachten}
\label{sec:opdrachten}

\subsection{Backup-script moderniseren}

Binnen Defensie gebruikt men een bashscript genaamd ``bu\_script.sh'' om een backup te nemen van de databanken die bestaan op een machine en deze weg te schrijven op het netwerk.  
Ik kreeg de opdracht om dit script te moderniseren. Om de opdracht een beetje te vereenvoudigen moest ik de backup's lokaal wegschrijven. Het script is te vinden in Bijlage \ref{sec:bu_script}

\subsubsection{Beginsituatie}

Het script bestaat maar het is verouderd. Met het script is het mogelijk om meerdere instances van Mariadb/mysql te backuppen. Het script is zeer lang en moeilijk leesbaar.

\subsubsection{Doel}

\begin{itemize}
    \item moderniseer het script
    \item maak het script meer flexibel door gebruik te maken van meer variabelen
    \item genereer een rapport van de uitvoering van het script
    \item schrijf de backup lokaal weg onder de map ``$\sim$/backup''
\end{itemize}

\subsubsection{Plan van Aanpak}

\begin{enumerate}
    \item lees het script door en probeer de denkwijze te begrijpen van de persoon die het script geschreven heeft
    \item verwijder delen die niet meer van toepassing zijn
    \item voeg meer commentaar toe aan het script zodat iemand die het script later zal doornemen sneller weet wat er gebeurd
    \item gebruik meer functies aan zodat het script meer leesbaar wordt
    \item test het script
\end{enumerate}

\subsubsection{Uitwerking}

Ik ben begonnen met het script door te nemen en de logica proberen te verstaan. Tijdens dat ik dit deed heb ik extra commentaar geschreven om duidelijk te maken wat er juist gebeurde in het script. Na een kort gesprek met Tom hebben we besloten om het script enkel te beperken voor server die maar 1 instance hebben van mariadb/mysql. Hierdoor werd het script al veel meer leesbaar. Door extra hulpfuncties toe te voegen is de leesbaarheid ook verbeterd. Tijdens het testen heb ik er nog enkele bugs uitgehaald maar het script werkt zoals gevraagd.

\subsubsection{Eindresultaat}

Het eindresultaat is een script dat makkelijker te lezen/verstaan en flexibel is. Het script gaat een rapport genereren van de uitvoering en de backup was te vinden in de map ``$\sim$/backup''. Er is mogelijkheid om in te toekomst het script uit te breiden zodat dit een melding gaat genereren in Zabbix (monitoring tool).

\subsubsection{Business doelstellingen}

Voor de business was het belangrijk dat het script een moderne update kreeg. Dit was zeker het geval.

\subsubsection{Persoonlijke doelstellingen}

Voor mij persoonlijk was dit een geslaagde opdracht. Het script is nu meer flexibel en dit zal er voor zorgen dat men dit in de toekomst nog kan uitbreiden.


\subsection{Procedure upgrade mariadb}

Tegen 2024 moeten alle dbserver binnen Defensie draaien op Mariadb 10.6 op RHEL 8. Om deze transitie rustig te laten verlopen is er nood aan een procedure die men kan volgen tijdens zo een upgrade van een machine.

\subsubsection{Beginsituatie}

De dbserver bij Defensie draaien op mariadb 5.5 op RHEL 7.

\subsubsection{Doel}

Schrijf een procedure die men kan volgen om een machine te upgraden naar Mariadb 10.6 op RHEL 8. De procedure moet aandacht schenken aan volgende punten:

\begin{itemize}
    \item het besturingssysteem van de nieuwe machine is Red Hat Enterprise Linux (RHEL) 8
    \item er gaat geen data verloren
    \item de juiste mensen inlichten dat er onderhoud zal doorgevoerd worden op hun dbserver
\end{itemize}

\subsubsection{Plan van Aanpak}

Voor deze opdracht zal ik iteratief werken:

\begin{itemize}
    \item de eerste iteratie zal ik op papier (in grote lijnen) schetsen wat er moet gebeuren
    \item de tweede iteratie zal ik dit in een word-document gieten en laten nalezen door Tom
    \item de derde en volgende iteratie(s) zal ik de procedure meer specifiek maken
    \item eens de procedure duidelijk is, ga ik te werk om bepaalde hulpscripts te schrijven. Deze moeten er voor zorgen dat de persoon die de transitie doet zo weinig mogelijk manueel werk heeft.
    \item schrijf het script om users te checken
    \item als laatste zal ik de procedure grondig testen.
\end{itemize}

\subsubsection{Uitwerking}

Nadat de procedure (zie Bijlage \ref{sec:upgrade-mariadb}) opgesteld was en nagekeken door Tom ben ik begonnen aan het script \verb*|check_users| (zie Bijlage \ref{sec:check-users}).

\subsubsection{Eindresultaat}
\subsubsection{Business doelstellingen}
\subsubsection{Persoonlijke doelstellingen}

\subsection{High availability bij databanken}


Onderzoek hoe men binnen defensie high availability implementeren? Als er een db faalt en niet beschikbaar is moet de applicatie zonder problemen verderwerken.

\subsubsection{Beginsituatie}

Binnen Defensie is er momenteel maar 1 databank achter een bepaalde applicatie. Dit zorgt er natuurlijk voor dat als die databank faalt, de applicatie ook niet meer zal werken.

\subsubsection{Doel}

Om de kans op het verliezen van data te verkleinen wil men bij Defensie databank replicatie implementeren.

\subsubsection{Plan van Aanpak}

Voor deze opdracht zal ik iteratief werken:

\begin{itemize}
    \item de eerste iteratie zal ik op papier (in grote lijnen) schetsen wat er moet gebeuren alsook ideeën opschrijven van productie die men mogelijk kan gebruiken
    \item de tweede iteratie zal ik dit in een word-document gieten en laten nalezen door Tom
    \item ik ga een master-slave opstelling testen
    \item ik ga proberen om de master-slave opstelling om te vormen naar een master-master
\end{itemize}

Nadat het document volledig is ga ik een handleiding maken om dit te implementeren en een eenvoudige opstelling testen met drie servers (twee \gls{db} servers en 1 een proxyserver).

\subsubsection{Uitwerking}

Ik heb een installatie handleiding (zie bijlage \ref{sec: replication-installation-guide}) gemaakt om een master-master opstelling op te zetten.

\subsubsection{Eindresultaat}

Er is een handleiding gemaakt die men kan volgen om master-master replicatie op te zetten voor een applicatie. Als er 1 server wegvalt dan zal de proxy automatisch schakelen tussen de \gls{db} servers.

\subsubsection{Business doelstellingen}

Er is replicatie tussen de databank servers.

\subsubsection{Persoonlijke doelstellingen}

Ik heb bijgeleerd over replicatie tussen databanken en hun verschillende opstellingen.


\subsection{MS SQL logins}

\subsubsection{Beginsituatie}

De Jboss Application servers connecteren zich met MSSQL server met SQL logins en dus via SQL server authentication.

\subsubsection{Doel}

Om het systeem veiliger te maken wil de dienst \gls{mssql} overschakelen naar Windows authenticatie bij het inloggen op een databank.

\subsubsection{Plan van Aanpak}

Ik ga beginnen met mezelf bekend te maken met het onderwerp. Ik ga een virtuele machine vragen waar ik op kan testen. Daarna zal ik een oplossing proberen zoeken op het probleem, en als laatste zal ik deze testen.

\subsubsection{Uitwerking}

Ik had een virtuele machine gevraagd die ik kon gebruiken als testomgeving. Blijkbaar is het niet zo simpel om voor mij een virtuele machine te voorzien dus hebben ze mij toegang gegeven op een bestaande databank. Zo had ik een omgeving om dingen te testen. Ik heb een gesprek gehad met Donovan om een oplossing te zoeken op het probleem.

\subsubsection{Eindresultaat}

Het is mogelijk om om windows authenticatie te gaan gebruiken voor de JBOSS applicaties, maar het is niet onbelangrijk om aandacht te schenken aan het volgende. Het wachtwoord van de gebruiker die je in \gls{ad} aanmaakt zal na een bepaalde tijd moeten veranderd worden (afgedwongen door de policy van \gls{ad}), dit is een grote administratieve taak als je dit moet doen voor alle servers. Om aan Windows authenticatie te doen kam men gebruik maken van dezelfde driver dat men nu ook gebruikt.

\subsubsection{Business doelstellingen}

Het systeem zal veiliger zijn met Windows Auhenticatie omdat er dan gebruikers gebruikt worden die gedefinieerd zijn in \gls{ad}.
    %%=============================================================================
%% Eindereflectie
%%=============================================================================
\section{Eindreflectie}
\label{sec:Eindreflectie}

\subsection{Voldoet de stage aan hetgeen ik verwacht had?}

Bij de aanvang van de stage was ik wat onwennig, omdat ik niet goed wist wat ik kon verwachten van een stage bij Defensie. Nu dit achter de rug is, vind ik het meer dan geslaagd. Ik ben terecht gekomen bij toffe mensen die me met open armen ontvangen hebben. Het was een unieke ervaring om te werken in een militaire sfeer.

\subsection{Overwelke opdracht ben ik het meest trots?}

De procedure om mariadb te upgraden, daar ben ik het meest trots op. Ik heb een script geschreven in Python, terwijl ik redelijk weinig ervaring heb met Python. Ook heb ik hier veel bijgeleerd, als je ook een upgrade van een besturingssysteem moet doen dan is de upgrade minder vanzelfsprekend.

\subsection{Is dit het toekomstbeeld dat ik voor ogen heb?}

Ja, werken met linuxservers zie ik me in de toekomst zeker nog doen, maar ik denk dat er ook andere dingen zijn die ik eens wil proberen zoals bijvoorbeeld \gls{ci}/\gls{cd} voorzien in een DevOps project.

\subsection{Vind je van jezelf dat je genoeg ervaring hebt opgedaan om nu in het werkveld te stappen?}

Zeker en vast, Ik heb veel bijgeleerd over linuxservers en het beheer ervan. Ook de provisioning van de server begrijp ik nu beter. 
Wanneer je bij een nieuwe werkgever start, vraagt het tijd voor jezelf in te werken en het beste uit jezelf te halen.
    \pagebreak
    
    \appendix
    % Set toc depth to only display sections
    \addtocontents{toc}{\setcounter{tocdepth}{1}}
    %%=============================================================================
%% Stage-logboek
%%=============================================================================

\section{Stagedagboek}
\label{sec:stagedagboek}

\subsection{Week 1}

\subsubsection{Maandag 21/02/2022}

\emph{Eerste stagedag}. Kennismaking met \textbf{Tom De Leeuw}, Tom
staat in voor het linuxgedeelte. Laptop ontvangen, vooral gepraat over mogelijkheden van de stage.

Werkwijze voor telewerken:

\begin{enumerate}
    \item Surf naar \url{https://portal.connect.mil.be} en meld aan met \textbf{itsme}
    \item Kies voor vpn. Eens de verbinding opstaat kan je beginnen met skype.
\end{enumerate}

\subsubsection{Dinsdag 22/02/2022}

werkuren: \emph{8:00 - 16:00}

Zelfstandig research gedaan naar \gls{lvm}. \url{https://access.redhat.com/documentation/en-us/red_hat_enterprise_linux/8/html/configuring_and_managing_logical_volumes/index}

Na de research enkele kleine opdrachtjes uitgevoerd op mijn eigen devserver binnen Defensie:

\begin{enumerate}
    \item aanmaken logical volume
    \begin{itemize}
        \item maak logical volume aan van 300mb
        \item mount logical volume op \texttt{/var/data01}
        \item herstart server en kijk of het nog bestaat.
    \end{itemize}
    \item aanpassen logical volume
    \begin{itemize}
        \item vergroot logical volume
        \item verklein logical volume
    \end{itemize}
    \item volume groups
    \begin{itemize}
        \item maak 2 volume groups van 100-200mb
    \end{itemize}
\end{enumerate}

Achteraf als afsluiter van de dag moest ik een grafische omgeving installeren op mijn devserver en proberen om via windows \gls{rdp} de server te kunnen overnemen.

\subsubsection{Woensdag 23/02/2022}

werkuren: \emph{8:00 - 16:00}

Zelfstandig research gedaan naar
\href{https://www.redhat.com/en/technologies/management/satellite}{Redhat
    Satellite}, \href{https://theforeman.org}{Foreman} en
\href{https://www.theforeman.org/plugins/katello/}{Katello}. Ook naar
filesecurity: \glspl{acl}, \gls{selinux}.

Verworven kennis toegepast in demo-omgeving. Samen met Tom de Satellite
omgeving ontdekt. Uitleg gekregen hoe alles werkt.

\subsubsection{Donderdag 24/02/2022}

werkuren: \emph{8:00 - 16:00}

Kennis opgefrist ivm Mariadb, mysql en appstreams in RHEL. Manueel
mariadb geïnstalleerd op devmachine, \gls{db} en users aangemaakt. Research
over puppet en aan de hand van puppet mariadb proberen installeren en
users en \glspl{db} aan te maken. Meeting bijgewoond met externe mensen.

\subsection{Week 2}

\subsubsection{Maandag 28/02/2022}

werkuren: \emph{08:00-16:00}

Afgewerkt waar ik vorige donderdag mee bezig was. Bezig aan het
moderniseren van het script \texttt{bu\_script.sh}.

\subsubsection{Dinsdag 01/03/2022}

werkuren: \emph{8:00 - 16:00}

\texttt{bu\_script.sh} afgewerkt. Eenvoudige zelfgeschreven versie.\\
Begin aan voorbereiding theorie voor donderdag.

\subsubsection{Woensdag 02/03/2022}

werkuren: \emph{8:00 - 16:00}

Voorbereiding voor Donderdag

\begin{itemize}
    \item research gedaan in verband met \href{https://www.zabbix.com/documentation/5.0/en/manual/introduction/about}{Zabbix} (monitoring tool)
    \item git kennis opgefrist
    \item toegang gekregen tot de gitlab servers van Defensie.
    \item hello world project geschreven in Puppet op mijn dev machine
\end{itemize}

\subsubsection{Donderdag 03/03/2022}

werkuren: \emph{8:00 - 16:00}

Introductie tot Zabbix en puppet door Donovan. Daarna zelf via puppet
zabbix proberen installeren op devmachine.

\subsection{Week 3}

\subsubsection{Maandag 07/03/2022}

werkuren: \emph{08:00-16:00}

\begin{itemize}
    \item denk oefening:    
    \begin{itemize}
        \item hoe kan je het \texttt{bu\_script.sh} uitbreiden zodat er een melding komt in Zabbix wanneer een backup faalt?
        \item zoek uit wat er misgegaan is met de devmachine (te weinig geheugen).
    \end{itemize}
\end{itemize}

\subsubsection{Dinsdag 08/03/2022}

werkuren: \emph{8:00 - 16:00}

Introductie in Jboss applicatie servers door Christophe. Christophe
heeft me alles goed uitgelegd en we hebben een testapplicatie gedeployed
op mijn devmachine. Ik heb (onder begeleiding van Christophe) een
applicatie mogen vernieuwen in de acceptatieomgeving. (netmanto)

\subsubsection{Woensdag 09/03/2022}

werkuren: \emph{8:00 - 16:00}

De introductie in Jboss Applicatie servers verder gezet en afgewerkt. We
hebben nu niet alles maar redelijk veel overlopen van jboss servers. Ook
is het nu duidelijk hoe deze gemonitord worden (Jboss Operations
Networks).

Een eerste stagegesprek.

\subsubsection{Donderdag 10/03/2022}

Werkuren: \emph{08:30 - 16:30}

Werkdag in Peutie.

De opdracht gekregen om een oplossing te zoeken voor volgende problemen:

\begin{itemize}
    \item hoe high availability implementeren voor een db die aan een applicatie hangt?
    \item maak een procedure om Mariadb 5.5 op RHEL 7.x to upgraden naar Mariadb 10.6 op RHEL 8 zonder verlies van gegevens.
\end{itemize}

Ook de toegangsbadge in orde gemaakt samen met Donovan.

\subsection{Week 4}

\subsubsection{Maandag 14/03/2022}

werkuren: \emph{08:00-16:00}

Verder gewerkt aan de denkoefeningen op een iteratieve manier. Alles
mooi in een word-documentje gegoten.

\subsubsection{Dinsdag 15/03/2022}

werkuren: \emph{8:00 - 16:00}

Verder gewerkt aan de denkoefeningen op een iteratieve manier. Er is
nood aan validatie van gebruikers en databanken. Hiervoor een concept
bedacht.

\subsubsection{Woensdag 16/03/2022}

werkuren: \emph{8:00 - 16:00}

Verder gewerkt aan de denkoefeningen.

Upgrade mariadb:

\begin{itemize}
    \item validatie voor users bedacht. gestart aan een script om dit uit te
    voeren maar ik zat vast gaandeweg dus nu ga ik eens kijken dat er geen
    andere manier is om dit te doen.
\end{itemize}

db high availability:

\begin{itemize}
    \item iteratie afgewerkt. Nu wachten op feedback.
\end{itemize}

Stageverslag: structuur opgemaakt.

\subsubsection{Donderdag 17/03/2022}

Werkuren: \emph{08:15 - 16:15}

Fysiek in Peutie.

Normaal was er een introductie van mssql gepland vandaag maar deze is
verzet naar een later moment. Ter vervanging heb ik verschillende
praktische oefeningen gedaan ivm LVM op mijn devmachine. Ook dit
allemaal besproken met Tom.

\subsection{Week 5}

\subsubsection{Maandag 21/03/2022}

werkuren: \emph{08:00-16:00}

Verder gewerkt aan de denkoefening om high availability te implementeren
bij databanken. Tom heeft voor mij een aantal virtuele machines laten
maken zodat ik hierop kan testen.

Stageverslag verder aangevuld.

\subsubsection{Dinsdag 22/03/2022}

Geen stage.\\
Jobevent van HoGent


\subsubsection{Woensdag 23/03/2022}

werkuren: \emph{8:00 - 16:00}

Verder gewerkt aan de denkoefening om high availability te implementeren
bij databanken: Testopstelling gemaakt.

\begin{itemize}
    \item
    databank master
    \item
    databank slave
    \item
    reverse proxy
\end{itemize}

Zie Figuur \ref{fig:netwerkdiagram-db-replicatie}, Dit is het netwerkdiagram van de opstelling.

\begin{figure}
    \centering
    \includegraphics{img/networkdiagram_db_replication.JPG}
    \caption{\label{fig:netwerkdiagram-db-replicatie}netwerk diagram}
\end{figure}

Er is nog uitbreiding mogelijk: Hoe van replica een master maken?

\subsubsection{Donderdag 24/03/2022}

Fysiek in Peutie

werkuren: \emph{8:00 - 16:00}

Introductie van \emph{mssql}\\
Onderzoeksopdracht besproken. ik zal hierover nog een document
ontvangen.\\
Tom heeft mij een rondleiding gegeven in de serverroom van in Peutie.

In de namiddag was er een drink binnen defensie waar de kolonel een
toespraak heeft gehouden. Het was tof om deze bij te wonen.

\subsection{Week 6}

\subsubsection{Maandag 28/03/2022}

werkuren: \emph{08:00-16:00}

Gewerkt aan de opdracht voor mssql. Vooral research gedaan en mezelf
bekend gemaakt met het onderwerp. Ook een virtuele machine gevraagd om dingen te testen.


\subsubsection{Dinsdag 29/03/2022}
\subsubsection{Woensdag 30/03/2022}
\subsubsection{Donderdag 31/03/2022}


\subsection{Week 7}
\subsection{Week 8}
\subsection{Week 9}
\subsection{Week 10}
\subsection{Week 11}
\subsection{Week 12}
\subsection{Week 13}
\subsection{Week 14}
    \pagebreak
    %%=============================================================================
%% Procedure Upgrade Mariadb
%%=============================================================================

\section{Procedure: Upgrade Mariadb}
\label{sec:upgrade-mariadb}

This procedure can be used to upgrade Mariadb 5.5 on RHEL 7 to Mariadb 10.6 on RHEL 8.

\begin{enumerate}
    \item Take vmware-snapshot of vm old db
    \item Deploy new vm
    \begin{itemize}
        \item OS: RHEL 8
        \item Puppet Modules: \verb*|mysqldb|
    \end{itemize}
    \item Validate users (script)
     \begin{itemize}
        \item Check if all existing users are defined in puppet
    \end{itemize}
    \item Validate databases (script)
    \begin{itemize}
        \item Check if all existing databases are defined in puppet + check parameters
    \end{itemize}
    \item Create databases and users with grants via puppet
    \item Plan a moment for the upgrade
    \item Notify the right people "DB will be under maintenance"
    \item Freeze the database
    \begin{itemize}
        \item Stop the application
        \item Edit firewall so no external connection can be made
    \end{itemize}
    \item Export data from old vm
    \item Import the data into new vm
    \item Notify the right people "db is ready to be used."
    \item If go (upgrade works)
    \begin{itemize}
        \item delete snapchot old vm
        \item delete old vm
    \end{itemize}
    \item If no go (upgrade failed)
    \begin{itemize}
        \item restore snapshot old vm
        \item delete snapshot old vm
        \item delete new vm
    \end{itemize}
\end{enumerate}
    \pagebreak
    %%=============================================================================
%% Install Mariadb Replication
%%=============================================================================

\section{MariaDB: Replication installation guide}
\label{sec: replication-installation-guide}

\subsection{Steps}

\subsubsection{Database replication}

\begin{enumerate}
    \item install MariaDB
    \item start and config MariaDB servers
    \item config firewall on servers
    \item test MariaDB
\end{enumerate}

\subsubsection{Reverse proxy}

\begin{enumerate}
    \item install proxy
    \item config proxy
    \item test proxy
\end{enumerate}

\subsection{Install MariaDB}

The repository is already there so we just have to install some packages.

\begin{lstlisting}
    sudo dnf install MariaDB-server MariaDB-backup
\end{lstlisting}

\subsection{Start/config first server}

Edit config file (\verb*|/etc/my.cnf|) with following:

\begin{lstlisting}
[mariadb]
log_bin
server_id=X
report_host=X
log-basename=masterX
binlog-format=mixed

log-slave-updates
auto_increment_increment=2
auto_increment_offset=X
\end{lstlisting}

\emph{\verb*|X| is 1 for the first master and 2 for the second master.}

Start the dbserver:

\begin{quote}
    If already running, restart dbserver to apply config.
\end{quote}

\begin{lstlisting}
    systemctl enable mariadb --now
\end{lstlisting}

There should be at least 1 user created for replication:

Create user accounts:

\begin{quote}
    Each account should be created on the master so replication can make these accounts on the slaves.
\end{quote}

\subsubsection{Replication user}

\begin{lstlisting}
    CREATE USER 'repl'@'serv_ip' IDENTIFIED BY 'test';
\end{lstlisting}

Grant required privileges.

\begin{lstlisting}
    GRANT REPLICATION SLAVE
    ON *.* TO repl@'serv_ip';
\end{lstlisting}

\subsection{Start/config second server}

\begin{lstlisting}
[mariadb]
log_bin
server_id=x
report_host=x
log-basename=masterX
binlog-format=mixed

log-slave-updates
auto_increment_increment=2
auto_increment_offset=X
\end{lstlisting}

\emph{\verb*|X| is 1 for the first master and 2 for the second master.}

\begin{lstlisting}
    SHOW MASTER STATUS;
    +--------------------+----------+--------------+------------------+
    | File               | Position | Binlog_Do_DB | Binlog_Ignore_DB |
    +--------------------+----------+--------------+------------------+
    | master1-bin.000096 |      568 |              |                  |
    +--------------------+----------+--------------+------------------+
\end{lstlisting}

Take notes of the filename and position

\begin{lstlisting}
    CHANGE MASTER TO
    MASTER_HOST='master_ip',
    MASTER_USER='replication_user',
    MASTER_PASSWORD='test',
    MASTER_PORT=3306,
    MASTER_LOG_FILE='master-bin.000096',
    MASTER_LOG_POS=568,
    MASTER_CONNECT_RETRY=10;
\end{lstlisting}

\begin{quote}
    With fresh master, you don't need to specify `MASTER\_LOG\_FILE` and `MASTER\_LOG\_POS`
\end{quote}

\subsubsection{Start replication}

\begin{lstlisting}
    START SLAVE;
    SHOW SLAVE STATUS;
\end{lstlisting}

If replication is running properly, both `Slave\_IO\_Running` and `Slave\_SQL\_Running` should be `Yes`.


\subsection{Test replication}

Use the following url to test replication: \url{https://mariadb.com/docs/deploy/topologies/primary-replica/enterprise-server-10-3/test-es/}

\subsection{Install proxy}

I chose to go with `haproxy`.\\
To install this proxy just use dnf.

\begin{lstlisting}
    sudo dnf install haproxy
\end{lstlisting}

\subsection{Config proxy}

\subsubsection{Edit configuration file}

\begin{lstlisting}
    # /etc/haproxy/haproxy.cfg
    
    defaults
    mode  tcp
    option mysql-check user haproxy_health
    frontend frontend
    # read-requests will arrive at port 3100
    bind *:3100
    # write-requests will arrive at port 3200
    bind *:3200
    # if write send to masters
    use_backend db_masters if { dst_port 3200 }
    # if read sent to slaves
    default_backend db_slaves
    
    backend db_masters
    server master1 10.8.131.120:3306
    server master2 10.8.131.121:3306
    
    backend db_slaves
\end{lstlisting}

\subsubsection{(re)start service}

\begin{lstlisting}
    sudo systemctl enable haproxy --now
\end{lstlisting}

\subsubsection{SELinux}

\begin{lstlisting}
    setsebool haproxy_connect_any 1
\end{lstlisting}

\subsection{Test proxy}

To test proxy, use  port 3200 te write data and port 3100 to query data.
    \pagebreak
    %%=============================================================================
%% Kleine opdrachtjes
%%=============================================================================

\section{Kleine opdrachtjes}

Tijdens mijn stage heb ik een aantal kleine opdrachtjes gekregen die niet echt passen tussen de grote opdrachten in sectie \ref{sec:opdrachten}. Daarom ga ik deze hier beschrijven.

\subsection{Schrijf een patch voor een databank}
\label{subsec:tables_patch}

In het tekstbestand \emph{output\_old.txt} en \emph{output\_new.txt} staan namen van tabellen in een databank. (hieronder is dit een dataset om te testen)

Het is de bedoeling om een bestand (\emph{patch\_command.txt}) te genereren met sql statements die alle tabellen die enkel in het bestand \emph{output\_old.txt} te vinden zijn te verwijderen.

De inhoud van het tekstbestand \emph{output\_old.txt} (testdata) is:

\lstinputlisting[numbers=left]{../scripts/tables_patch/output_old.txt}

De inhoud van het tekstbestand \emph{output\_new.txt} (testdata) is:

\lstinputlisting[numbers=left]{../scripts/tables_patch/output_new.txt}

De inhoud van het bestand dat we gegenereerd hebben (\emph{patch\_command.txt}) is:

\lstinputlisting[numbers=left, language=sql]{../scripts/tables_patch/patch_command.txt}

\subsubsection{Script}

\lstinputlisting[numbers=left, language=bash]{../scripts/tables_patch/tables_patch.sh}
    \pagebreak
    %%=============================================================================
%% BU script
%%=============================================================================
\section{bu\_script}
\label{sec:bu_script}

\lstinputlisting[numbers=left, language=Bash]{../scripts/bu\_script/new\_bu\_script.sh}
    
    % Verklarende woordenlijst
    \pagebreak
    \printglossary[title=Verklarende woordenlijst, toctitle=Verklarende woordenlijst]
    
    
    %Referentielijst
    \pagebreak
    \printbibliography[heading=bibintoc]
    
    
\end{document}