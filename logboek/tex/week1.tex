
\hypertarget{stage-logboek-week-1-21022022---24022022}{%
\section{Stage logboek: Week 1 (21/02/2022 -
24/02/2022)}\label{stage-logboek-week-1-21022022---24022022}}

\hypertarget{maandag-21022022}{%
\subsection{Maandag 21/02/2022}\label{maandag-21022022}}

\emph{Eerste stagedag}. Kennismaking met \textbf{Tom De Leeuw}, Tom
staat in voor het linuxgedeelte. Laptop ontvangen, vooral gepraat over
mogelijkheden van de stage.

Werkwijze voor telewerken:

\begin{enumerate}
\def\labelenumi{\arabic{enumi}.}
\tightlist
\item
  surf naar \url{https://portal.connect.mil.be} en meld aan met
  \textbf{itsme}
\item
  Kies voor vpn. Eens de verbinding opstaat kan je beginnen met skype.
\end{enumerate}

\hypertarget{dinsdag-22022022}{%
\subsection{Dinsdag 22/02/2022}\label{dinsdag-22022022}}

werkuren: \emph{8:00 - 16:00}

Zelfstandig research gedaan naar LVM (Linux Volume manager)
--\textgreater{}
\url{https://access.redhat.com/documentation/en-us/red_hat_enterprise_linux/8/html/configuring_and_managing_logical_volumes/index}
Na de research enkele kleine opdrachtjes uitgevoerd op mijn eigen
devserver binnen Defensie:

\begin{enumerate}
\def\labelenumi{\arabic{enumi}.}
\tightlist
\item
  Aanmaken logical volume

  \begin{itemize}
  \tightlist
  \item
    Maak logical volume aan van 300mb
  \item
    mount logical volume op \texttt{/var/data01}
  \item
    herstart server en kijk of het nog bestaat.
  \end{itemize}
\item
  Aanpassen logical volume

  \begin{itemize}
  \tightlist
  \item
    vergroot logical volume
  \item
    verklein logical volume
  \end{itemize}
\item
  Volume groups

  \begin{itemize}
  \tightlist
  \item
    maak 2 volume groups van 100-200mb
  \end{itemize}
\end{enumerate}

Achteraf als afsluiter van de dag moest ik een grafische omgeving
installeren op mijn devserver en proberen om via windows rdp de server
te kunnen overnemen.

\hypertarget{woensdag-23022022}{%
\subsection{Woensdag 23/02/2022}\label{woensdag-23022022}}

werkuren: \emph{8:00 - 16:00}

Zelfstandig research gedaan naar
\href{https://www.redhat.com/en/technologies/management/satellite}{Redhat
Satellite}, \href{https://theforeman.org}{Foreman} en
\href{https://www.theforeman.org/plugins/katello/}{Katello}. Ook naar
filesecurity: ACL's, SELinux

Verworven kennis toegepast in demo-omgeving. Samen met Tom de Satellite
omgeving ontdekt. uitleg gekregen hoe alles werkt.

\hypertarget{donderdag-24022022}{%
\subsection{Donderdag 24/02/2022}\label{donderdag-24022022}}

werkuren: \emph{8:00 - 16:00}

Kennis opgefrist ivm Mariadb, mysql en appstreams in RHEL. Manueel
mariadb geïnstalleerd op devmachine, db en users aangemaakt. Research
over puppet en aan de hand van puppet mariadb proberen installeren en
users en db's aan te maken. Meeting bijgewoond met externe mensen.

